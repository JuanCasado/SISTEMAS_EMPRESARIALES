%Preambulo
\documentclass[12pt,letterpaper]{article}

\usepackage[utf8]{inputenc}
\usepackage{lmodern,textcomp}
\usepackage[spanish]{babel}
\usepackage{amsmath,amsfonts,amssymb}
\usepackage{graphicx}
\usepackage{multirow}
\bibliographystyle{plain}

\title{Estudio Técnico de Alternativas TIC \\ Henarsecur}
\author{David Menoyo Ros - 54005758X\\
		David Márquez Mínguez - 47319570Z\\
		Juan Casado Ballesteros - 09108762A\\
		Álvaro Vaya Arboledas - 09101090J\\
		}
\date{5 de marzo de 2019}


%Cuerpo
\begin{document}
	\maketitle %Me añade al inicio del documento
			   %la info de title, aithor e 
			   %incluso fecha de realización
			   
	\newpage
	\section{Contexto de Necesidad}
	Actualmente la empresa Henarsecur se encuentra decidida en implementar las TIC necesarias que permitan la expansión, modernización y optimización del negocio.\par
	Dicha necesidad queda más que reflejada en el estudio minucioso de las necesidades TIC de la empresa, por lo que el hecho de implementarlo o no no es objeto de debate, pero sí el método que se debe aplicar.\par
	Actualmente, Henarsecur no goza de absolutamente ninguna infraestructura TIC sofisticada, por lo que partimos de una gran oportunidad para implementar desde el principio el sistema que más ventajas aporten al negocio. Por lo que teniendo en cuenta el estado actual de la compañía, y los objetivos a medio-largo plazo, consideramos que la implementación de un sistema informatizado del negocio tiene que estar enfocado en una de las dos siguientes direcciones:
	\begin{enumerate}
	\item Infraestructura Web y de Red Local
	\item Infraestructura Web y de Red Externalizada
	\end{enumerate}
	
	\section{Propuestas}
	\subsection{Infraestructura Web y de Red Local}
	
	Proponemos la implementación de un sistema de red local en el que todos los equipos informáticos de la empresa se encuentren comunicados en LAN(Red de Área Local).\\
	Además, aprovechando la iniciativa de desarrollar una red local en la compañía, se propone a su vez la creación de un servidor web propio el cual funcionase bajo la infraestructura informática de Henarsecur.\par

De esta manera, se conseguiría crear un sistema totalmente centralizado y gestionado por la empresa, en donde se tendría control y responsabilidad absoluta de las acciones que se llevaran a cabo. No obstante, vamos a comentar las posibles ventajas e inconvenientes:
	\subsubsection{Ventajas} 
	Dicho sistema permite:
	\begin{itemize}  
	\item Gestionar y controlar todo lo que ocurre en nuestro sistema de red local, tanto en la comunicación entre nuestros dispositivos informáticos como en el servidor web.
	\item Capacidad de respuesta rápida ante cualquier incidencia puesto que la cercanía y el conocimiento de la infraestructura por parte de los empleados permitirían gozar de una gran calidad en la gestión y mantenimiento.\par
	Actualmente la empresa dispone de personal cualificado capaz de realizar dichas tareas de control y gestión de la infraestructura de red, no obstante, en caso de que se solicitara, se podría ofrecer un curso para profundizar o introducir las bases del puesto de administrador de infraestructuras de red y web a aquellos empleados que lo solicitaran.
	\end{itemize}
	
	\subsubsection{Inconvenientes}
	Dicho sistema requeriría:
	\begin{itemize}  
	\item Gastar recursos materiales, temporales y de personal en el correcto funcionamiento del mismo.
	\item Invertir inicialmente una gran cantidad de dinero en establecer la arquitectura de red, lo cual conllevaría:
	\begin{itemize}  
	\item Adquirir todo el hardware necesario para el correcto funcionamiento del sistema(ordenadores, discos duros, servidores, cableado, etc.).
	\item Desarrollar la página web, de manera que una vez finalizada, se implementaría en el servidor local de la empresa y se abriría a Internet para que los clientes pudieran acceder.\par El desarrollo de la página web estaría a cargo de COMPUTER CONSULTING S.L. 
	\end{itemize}
	\item La realización de un curso de iniciación o profundización en la gestión de una infraestructura red y web supondría un coste.
	
	\item En un mundo tan dinámico como el actual, puede costar mucho esfuerzo actualizar los sistemas e infraestructuras propias de cada empresa, de manera que es más probable que se queden obsoletos con el tiempo.
	
	\item Salvo que se hagan esfuerzos para evitarlo, un sistema propio  es más probable que presente vulnerabilidades de seguridad, sobre todo en ataues de fuerza bruta o de denegación de servicio.
	\end{itemize}
	
	\subsubsection{Resumen y Presupuesto}
	La implementación de una red propia y de un servidor propio aportaría una considerable ventaja en el control y mantenimiento del sistema informático, pero las vulnerabilidades en seguridad y la problemática en cuanto a los recursos necesarios para mantener dicha infraestructura podrían lastrar todas las ventajas obtenidas.\par
	El presupuesto contemplado es el siguiente:
	\begin{center}
    \begin{tabular}{ | l | l | p{7cm} |}
    \hline
    Componente & Coste & Observaciones \\ \hline
    Componentes Hardware & 9900€& Cableado, servidores, periféricos, routers, etc. \\ \hline
    Mano de obra & 3500€ & Instalación y puesta a punto. \\ \hline
    Curso a empleados & 350€ & Instalación y puesta a punto. \\ \hline
    \end{tabular}
	\end{center}
	
	\newpage
	
	
	\subsection{Infraestructura Web y de Red Externalizada}
	Proponemos la externalización de prácticamente la totalidad del sistema web, al igual que gran parte de las tecnologías TIC de la empresa, de manera que a cambio de una cuantía mensual/anual, un tercero se encargaría de toda la problemática que una página web y una infraestructura informática conlleva.\\
	Por tanto, los aparatos que Henarsecur tendría que disponer en su posesión se limita a una muy buena conexión a Internet y a unos periféricos que interactúen con los sistemas en la nube.
	\subsubsection{Ventajas} 
	Dicho sistema permite:
	\begin{itemize}  
	\item Abstraer toda la conplegidad subyacente a la gestión, mantenimiento y actualización de todo sistema informático a simplemente el pago de una cuantía mensual.
	\item Gran capacidad de actualización y adaptación a nuevas circunstancias, ya que con solamente pagar dinero se pueden conseguir cosas como:
	\begin{itemize}  
	\item Actualizar los sistemas informáticos a versiones más modernas casi al instante.
	\item Recibir asesoramiento continuo y profesional acerca del funcionamiento de las tecnologías.
	\item Gran capacidad de optimización de los recursos, ya que podemos contratar el plan que beneficie a la empresa sin tener apenas que gastar dinero innecesariamente.
	\item Gran capacidad de extensión de las funcionalidades, puesto que esto solo requiere contratar dicha nueva funcionalidad.
	\item Seguridad informática en cuento a ciberataques, robos de información o caídas/fallos informáticos, ya que las empresas que gestionan nuestras tecnologías están extremadamente preparadas para responder ante cualquiera de estos problemas.
	
	\end{itemize}
	\end{itemize}
	
	\subsubsection{Inconvenientes}
	Podemos destacar los siguientes inconvenientes:
	\begin{itemize}  
	\item Si bien actualmente existe un amplio abanico de planes y funcionalidades de externalización en el mercado, siempre estaremos restringidos a lo que el sector nos ofrezca.
	\item Necesidad de realizar una muy buena comunicación entre la empresa subcontratada y la empresa principal, de manera que se intente evitar la mayor cantidad de problemas y malentendidos posibles
	
	\end{itemize}
	
	\subsubsection{Resumen y Presupuesto}
	La externalización de gran parte de nuestro tejido TIC puede garantizarnos estar siempre a la vanguardia, lo cual nos arroje muchas ventajas competitivas tanto en gestión interna como en trato con los clientes( página web por ejemplo), pero todo ello a costa de ceder cierto grado de control sobre nuestros servicios.\par
	El presupuesto contemplado es el siguiente:
	\begin{center}
    \begin{tabular}{ | l | l | p{7cm} |}
    \hline
    Componente & Coste & Observaciones \\ \hline
    Externalización & 1550€/mes & Plan adecuado a las necesidades actuales de la empresa \\ \hline
    Hardware propio & 2300€ & Ordenadores y periféricos propios de la empresa. \\ \hline
    \end{tabular}
	\end{center}
	
	\section{Conclusión y Propuesta Elegida}
	COMPUTER CONSULTING S.L. aconseja implementar la segunda opción puesto que dadas las necesidades reales de la Henarsecur, consideramos que lo más conveniente es tener un sistema flexible, seguro y profesional, de manera que aunque a la larga el coste que conlleva la externalización se pudiera incrementar, se vería compensado por el ahorro de la inversión inicial, y por la extremada optimización de los recursos TIC.


	\section{Propuestas Finales}
	Para poder realizar un outsourcing efectivo tenemos que elegir el
	el ARP correcto, de manera que elijamos inteligentemente los
	modulos que integrar o no en función de las necesidades de la empresa.\\ Por ejemplo: 		  \textbf{Gestión financiera, ventas, CRM y RRHH}
	
	
	 En cambio, módulos como de fabricación no serían tan necesarios, y por tanto podríamos ahorrar muchos gastos innecesarios.
	Con el ARP obtendrán una base de datos centralizada a partir de un programam único , de modo que todos sus departamentos quedarán enlazados, y por tanto podrán compartir la información que actualmente está aislada. Además, se solventará el problema de comunicación entre los departamentos, reduciendo el uso de técnicas poco eficientes como el uso de papel.
	
	Prime
	Netsuite(Oracle).
	\begin{itemize}%Ponerlo en negrita( resaltar)  
	\item \textbf{Establecer los resultados que queremos obtener con el ARP:}
	  \begin{itemize}%Ponerlo en negrita( resaltar)  
	  \item Mejorar la comunicación dentro de la empresa.
		
	
	  \end{itemize} 
	\end{itemize} 
	
	
\end{document}
		
